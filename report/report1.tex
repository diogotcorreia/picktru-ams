\documentclass[12pt,a4paper]{article}
\usepackage[legalpaper, portrait, lmargin=1cm, rmargin=1cm, tmargin=2cm, bmargin=2cm]{geometry}
\usepackage{fancyhdr}
\usepackage{amsmath}
\usepackage{amssymb}
\usepackage{graphicx}
\usepackage{wrapfig}
\usepackage{blindtext}
\usepackage{hyperref}
\usepackage{pdflscape}
\usepackage{svg}
\usepackage[most]{tcolorbox}

\graphicspath{ {./} }
\hypersetup{
  colorlinks=true,
  linkcolor=blue,
  filecolor=magenta,
  urlcolor=blue,
  citecolor=blue,
  pdftitle={Relatório AMS - Entrega 1 - 2022/2023},
  pdfpagemode=FullScreen,
}

\definecolor{bg}{rgb}{1,0.96,0.9}

\pagestyle{fancy}
\fancyhf{}
\rhead{Grupo \textbf{37}}
\lhead{Relatório Entrega 1 AMS 2022/2023 LEIC-A}
\cfoot{Diogo Gaspar (99207), Diogo Correia (99211) e Tiago Silva (99335)}

\renewcommand{\footrulewidth}{0.2pt}

\renewcommand{\labelitemii}{$\circ$}
\renewcommand{\labelitemiii}{$\diamond$}


\begin{document}
\begin{titlepage}
  \begin{center}
    \vspace*{5cm}

    \Huge
    \textbf{Projeto AMS - Entrega 1}

    \vspace{0.5cm}
    \LARGE
    Grupo 037 | Turno L08 | LEIC-A

    \vspace{0.5cm}
    \large
    Prof. Maria do Rosário Carvalho

    \vfill
  \end{center}
  \large
  \begin{itemize}
    \item[] \textbf{Diogo Gaspar} (99207) - 14h
    \item[] \textbf{Diogo Correia} (99211) - 14h
    \item[] \textbf{Tiago Silva} (99335) - 14h
  \end{itemize}
\end{titlepage}

\begin{tcolorbox}[enhanced jigsaw,colback=bg,boxrule=0pt,arc=4pt]
  \begin{large}
    \textbf{Notas:}
  \end{large}

  \begin{small}
    \textbf{Diagrama de Vista Geral do Negócio}
  \end{small}
  \begin{itemize}
    \item Optou-se por não se explicitar todos os tipos de relatórios resultantes do processo P-ON,
          considerando apenas um tipo \textit{"Relatório"} genérico, dado que se considerou tal
          divisão como um detalhe interno do processo.
  \end{itemize}

  \begin{small}
    \textbf{Diagrama do Processo P-SET}
  \end{small}
  \begin{itemize}
    \item Utilizou-se o \textit{Data Object} "Plano de Projeto", dado ser um documento
          utilizado frequentemente durante o decorrer do processo.
          Não existe, porém, qualquer informação relativa ao mesmo ter de ser guardado permanentemente/acedido
          por outros processos, pelo que a opção \textit{Data Store} foi prontamente descartada.
  \end{itemize}

  \begin{small}
    \textbf{Diagrama do Processo P-ON}
  \end{small}
  \begin{itemize}
    \item Utilizaram-se \textit{Parallel Gateways} sempre que um fluxo teria de esperar por uma
          mensagem após a execução de uma tarefa, de forma a prevenir problemas de sincronização,
          tal como indicado \href{https://moodle.dei.tecnico.ulisboa.pt/mod/forum/discuss.php?d=7213}{nesta questão no Moodle da UC}.
    \item Optou-se pela utilização de \textit{Message Start Events} em vez de \textit{Signal Start Events}, dado que nada é dito
          no Universo de Discurso sobre quem contacta a U-HW e a U-SW.
    \item Considerou-se que a U-HW não espera que a U-SW termine de analisar o plano
          de trabalhos para se deslocar até à loja, dado que tal não é explícito no Universo de Discurso.
  \end{itemize}
\end{tcolorbox}

\begin{landscape}
  \begin{center}
    \includesvg[inkscapelatex=false,width=1.5\textwidth]{assets/archimate.svg}
  \end{center}
\end{landscape}

\begin{landscape}
  \begin{center}
    \includesvg[inkscapelatex=false,width=1.5\textwidth]{assets/p_set.svg}
  \end{center}
\end{landscape}

\begin{landscape}
  \begin{center}
    \includesvg[inkscapelatex=false,width=1.35\textwidth]{assets/p_on.svg}
  \end{center}
\end{landscape}

\end{document}